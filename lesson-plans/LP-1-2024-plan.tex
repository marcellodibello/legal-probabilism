%\documentclass[11pt]{book}
\documentclass[16pt]{article}
%\documentclass[11pt]{amsart}
%\documentclass[preprint,12pt]{elsarticle}
\usepackage{multirow}
\usepackage[all]{xy}
\usepackage{verbatim} 
\usepackage{endnotes}
\usepackage{amssymb} 
\usepackage{setspace} 
%\doublespacing
%\onehalfspacing
%\usepackage{mathabx}
\usepackage{amsmath}

\usepackage{footmisc}
\renewcommand{\footnotelayout}{\doublespacing}

%\usepackage{diagrams}

%\usepackage{bookman}
%\usepackage{helvet}
%\usepackage{palatino}
%\usepackage{times}


%\usepackage{charter}
%\usepackage{avant}
%\usepackage{chancery}
%\usepackage{utopia}

\def\setgrouptext#1{\gdef\grouptext{#1}}
\newenvironment{groupeditems}{\begin{displaymath}\left.\vbox\bgroup\setgrouptext}{%
  \egroup\right\rbrace\hbox{\grouptext}\end{displaymath}}
  
\usepackage{fancyhdr}

%\pagestyle{fancy}

\usepackage{sectsty}
%\allsectionsfont{\sffamily} 
\chapterfont{\Large \scshape}
\sectionfont{\large \scshape }
\subsectionfont{\normalsize \scshape}
%\subsectionfont{\large \nohang \flushleft} 
%\allsectionsfont{\mdseries\itshape} 
%\sectionfont{\fontfamily{ptm}\selectfont}

 %\usepackage{fullpage}

\usepackage[margin=1in]{geometry}
\usepackage{lastpage}

\usepackage{fancyhdr}
\setlength{\headheight}{15.2pt}
\pagestyle{fancy}

%\rhead[<even output>]{\textsc{\large Dissertation Summary} --  \thepage \textit{ of}   \pageref{LastPage}}
\rhead[<even output>]{\textsc{\small  Legal Probabilism} -- \ \thepage \textit{ of}   \pageref{LastPage}}
%\chead[<even output>]{<odd output>}
\lhead[<even output>]{\text{\small }}

%\lfoot[<even output>]{<odd output>}
\cfoot[<even output>]{}
%\rfoot[<even output>]{ \thepage \textit{ of}   \pageref{LastPage}}




\usepackage{pgfplots}


\pgfmathdeclarefunction{gauss}{2}{%
  \pgfmathparse{1/(#2*sqrt(2*pi))*exp(-((x-#1)^2)/(2*#2^2))}%
 }
 
\usepackage[round]{natbib}
\begin{document}

%\noindent
%\textsc{\large \bf Lottery propositions, bracketing, and epistemic closure.}

\thispagestyle{empty}

\vspace{-2cm}
\noindent
\section*{\Large Legal Probabilism} 
\subsection*{Marcello Di Bello -- ASU}
\subsection*{Lesson plan \#1 -- August 22, 2024}

\vspace{5mm}


\begin{enumerate}
\item Rough defintion of legal probabilism from the SEP entry
\item Probability and its many applications (gambling, but also stock market, policy, evidence-based medicine). 
Why not apply it to legal decision-making as well?
\item Motivating questions for the course: 
\begin{enumerate}
\item Evidence weighing: Can we weigh evidence probabilistically and say it is 90\% likely the defendant is guilty?
\item Decision-making: Is BARD a probability threshold?
\item System analysis: Can probability help us understand and design a better trial system and justice system overall?
\end{enumerate}
\item Need to learn a but of probability theory. Drive home the point that probability is a difficult notion to 
master, and we can easily misunderstand it or make mistakes as we reason probabilistically.
\begin{enumerate}
\item  Introduce example of three cards: RED-RED, WHITE-WHITE, RED-WHITE. If you draw one card at random from the three and its side is RED (we do not know about the other side), what is the probability that the card being drawn is a RED-RED card? Ask. 
\item One common answer is one-half, because, one can reason, there were three cards, the WHITE-WHITE card is out, so there are two cards left, so the probability of one half. 
\item Actually, the probability that it is a RED-RED card is two third if one lists all the possibilities. 
\item This stresses the importance of carving out the space of possibilities whenever you are estimating the probability of something.
\end{enumerate}
%
\item Interpretation of probability.
\begin{enumerate}
\item Classical or naive. Cube factory problem here.
\item Frequency. But does it apply to legal proceedings?
\item Epistemic. Is it really appropriate to measure the strength and weakness of the evidence? Problem of negation.
\item Model based. Example is earthquake in California.
\end{enumerate}
%
\item Mathematics of probability
\begin{enumerate}
\item The Kolmogorov axioms
\item The negation lemma as an example of something you can prove from the axioms
\item Williamson example showing one cannot use the axioms unproblematically. 
\end{enumerate}
%
\item Second part of the class, talk about the Collins case.
\begin{enumerate}
\item One issue to note here is whether the statistical estimates were correct or well-justified. Clearly, they were not. Also, the application of the product rule here was completely mistaken.
\item But there is a further issues, which goes back to the coincidence argument. Suppose the 1 in 12 million estimate is good. Since Collins match the description and such a description is so rare that only 1 in 12 million couples in California would match it, can't we conclude that the probabilistic/statistical identification was good? Wouldn't it be a colossal coincidence if someone else was the culprit other than the Collins?
\item Introduce the prosecutor fallacy, also called inversion fallacy. The probability 1 in 12 million is the probability that a random couple would match the description, that is

\[P(\textit{match description}| \textit{random couple}).\]

 If we replace `random couple' with 'innocent couple', we get that 1 in 12 million is the probability that an innocent couple would match the description, that is
 
 \[P(\textit{match description}| \textit{random couple}).\]
 
  But this is the inverse (hence the name, inversion fallacy) of the probability that the Collins were innocent given that they matched the description, that is
  
  \[P(\textit{innocent couple}| \textit{match}).\]

  
 The prosecutor fallacy consists in confusing or inverting  $P(A| B)$ and $P(B|A)$.
 
 \end{enumerate}
\item But drive home a more problematic point. Sure, errors are committed and should be avoided. But there is also a philosophical/legal question which we have not yet neither asked nor answered. Why was guilt not be proven beyond a reasonable doubt in the Collins case (assuming the statistics were good, which they not)? We cannot simply say  mathematical?statistical proof does not satisfy a proof beyond a reasonable doubt. 

\item Suppose that instead of a statistical identification, we had an eyewitness who pointed at the Collins (just because they matched the witness internal, unexpressed description). If the witness survived cross-examination, wouldn't we say that guilt was proven beyond a reason doubt? Are't we victim of some cognitive illusion in being so hesitant in convicting on the basis of mathematical proof?

\item Maybe, to show the ambiguity of the criminal standard of proof conclude with (or begin with) Beccaria's quotation.
\end{enumerate}


\end{document}
